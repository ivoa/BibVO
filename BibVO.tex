\documentclass[11pt,a4paper]{ivoa}
\input tthdefs

\definecolor{termcolor}{rgb}{0.6,0.1,0.1}

\iftth
\def\vocterm#1{\emph{\color{termcolor}#1}}

\else
\def\vocterm{\startvocterm\realvocterm}
\def\realvocterm#1{\emph{\color{termcolor}#1}\endvocterm}
\begingroup
\gdef\breakablecolon{:\hskip0pt}
\catcode`\:=\active
\gdef\startvocterm{\begingroup
  \catcode`\:=\active\let:=\breakablecolon}
\gdef\endvocterm{\endgroup}
\endgroup
\fi

\lstloadlanguages{sh,SQL}
\lstset{flexiblecolumns=true,basicstyle=\ttfamily,showstringspaces=False}

\title{Bibliographic Interfaces in the Virtual Observatory}

% see ivoatexDoc for what group names to use here; use \ivoagroup[IG] for
% interest groups.
\ivoagroup{Data Curation and Preservation}

\author[http://www.ivoa.net/cgi-bin/twiki/bin/view/IVOA/MarkusDemleitner]{Demleitner, M.}
\author{Accomazzi, A.}

\editor{Demleitner, M.}

% \previousversion[????URL????]{????Concise Document Label????}
\previousversion{This is the first public release}


\begin{document}
\begin{abstract}
This note documents existing approaches to interfacing the Virtual
Observatory to bibliographic services, in particular the Astrophysics
Data Service ADS, and suggests amendments to these existing practices
where necessary.
\end{abstract}


\section*{Conformance-related definitions}

The words ``MUST'', ``SHALL'', ``SHOULD'', ``MAY'', ``RECOMMENDED'', and
``OPTIONAL'' (in upper or lower case) used in this document are to be
interpreted as described in IETF standard RFC2119 \citep{std:RFC2119}.

The \emph{Virtual Observatory (VO)} is a
general term for a collection of federated resources that can be used
to conduct astronomical research, education, and outreach.
The \href{https://www.ivoa.net}{International
Virtual Observatory Alliance (IVOA)} is a global
collaboration of separately funded projects to develop standards and
infrastructure that enable VO applications.

Items written \vorent{like this} in the following refer to items from
the VOResource data model.

\section{Introduction}

In addition to ``blind'' data discovery within the VO, users frequently
want to know about VO resources in their relationship to the wider
world, in particular that of scholarly publications.  Use cases we
envision are, specifically,

\begin{itemize}
\item A user of a bibliographic service has discovered a publication of
interest; they should also be directed to a VO resource that supplements
that publication.

\item A user of a bibliographic service has discovered a publication
that has used VO datasets; they should be notified of the availability
of these datasets in standards-compliant data services.

\item A user of a VO resource wants to properly give credit; we
argue that resources published as a supplement to a scholarly
publication need a different treatment here from resources not directly
derived from one.
\end{itemize}

\section{Linking VO Resources to Articles}

The VOResource metadata schema \citep{2018ivoa.spec.0625P} contains the
\vorent{content/source} field, defined as

\begin{quotation}
\noindent A bibliographic reference from which the present resource is
derived or extracted.

\noindent This is intended to point to an article in the published
literature. An ADS Bibcode is recommended as a value when available.
\end{quotation}

\noindent in VOResource 1.2.  VOResource 1.3 will presumably add DOIs as a
recommended identifier type.

It is generally understood that, absent resource-specific regulations
communicated in \vorent{rights}, users of a VO services should cite the
referenced article when a resource has played a significant role in some
published research.  Facilitating this citation is one major reason for
having machine-readable identifiers in that field.  Care must be taken
to properly declare the \vorent{source/@format} to either \verb|bibcode|
or \verb|doi|. Otherwise, the query below will miss useful metadata.

Conversely, bibliographic services can use this information to enrich
bibliographic records with links to data sources (``D links'' in ADS
nomenclature).  To retrieve pairs of ivoid and bibcode, the following
RegTAP \citep{2019ivoa.spec.1011D} query can be used:

\begin{lstlisting}[language=SQL]
select ivoid, source_value, source_format
from rr.resource
where source_format in ('doi', 'bibcode')
\end{lstlisting}

This query will work on any modern searchable registry\footnote{A list
of searchable VO registries can be obtained from the Registry of
Registries at \url{https://rofr.ivoa.net}.}.  Note, however, that
default match limits of the TAP services may already truncate this
response, so TAP clients should take care to pass a suitable large
MAXREC.

Without VO tooling, a script like the following will yield this data in
easily consumable CSV form:

\lstinputlisting[language=sh]{harvest.sh}

\section{Linking Datasets To Articles}

The second scenario is that resources within a data centre are used or
originate in some publication and the data centre operators want to
notify bibliographic services of that fact.  Again, this might yield the
equivalent of ADS ``D links''.  A prototype example here is an SSAP
\citep{2012ivoa.spec.0210T} service that has a non-constant
\textsl{meta.bib.bibcode} column.

Bibliographic services \emph{could} try to harvest such information from
the services themselves.  However, even where there is a standard
location as in SSAP, the protocols are usually not well suited for
harvesting, and it would certainly be unattractive for bibliographic
services to have to harvest dozens or hundreds of VO services with
something like weekly cadence.

\subsection{The biblink-harvest endpoint}

Therefore, we propose a new endpoint; a data centre could have one or a
few of these for all its data holdings.  This endpoint is a GET-able
resource returning a application/json-encoded sequence of link records.
Each link record MUST have the following keys:

\begin{itemize}
\item \emph{bib-ref} -- a machine-readable identifier of a bibliographic
source; currently, either a bibcode or a DOI.
\item \emph{relationship} -- a relationship identifier; see below.
\item \emph{dataset-ref} -- a reference to the dataset or VO resource cited.
This SHOULD be a URI understood by common web browsers; what is referenced
does not need to be renderable by them (e.g., it could be a FITS file).
\end{itemize}

Optionally, a key \emph{bib-format} can be given.  If it is missing,
harvesters may assume bibcodes in bib-ref.  Otherwise, it should contain
one of the values defined for \vorent{source/@format} in VOResource.

Optionally, a key \emph{anchor-text} can be given.  This contains a
descriptive, short text that the bibliographic service should use as
the anchor text of the link element.

Optionally, a key \emph{cardinality} can be given.  This is when the
data centre does not actually report individual links but rather a link
to a centre-side page keeping per-publication links.

All other keys are reserved except those beginning with an underscore
character.  These can be used to prototype extensions or -- not
encouraged -- to communicate extra metadata between specific data
centres and specific bibliographic services.

Since we expect responses from such endpoints to be always of manageable
size –– a few megabytes --, we do not define a harvesting mode for this
type of endpoint for now.

TODO: examples

\subsection{Relationship Types}

The \emph{relationship} field should be an identifier from the IVOA
relationship\_type\footnote{\url{http://ivoa.net/rdf/relationship_type}}
vocabulary.  Typical terms data centres would use are

\begin{itemize}
\item \vocterm{Cites} is used when \emph{bib-ref} derives results from
\emph{dataset-ref}.
\item \vocterm{IsSupplementedBy} is used when data derived in
\emph{bib-ref} is published in \emph{dataset-ref}.
\end{itemize}

As a guideline, the link record can be interpreted as an RDF triple
(\emph{bib-ref}, \emph{relationship}, \emph{dataset-ref}).


\subsection{Discovering biblink-harvest endpoints}

Data centres register their biblink-harvest endpoints as VODataService
\citep{2021ivoa.spec.1102D} DataService-s.  These must have a capability
with a \verb|standardID| of (for now; this will probably change if this
endpoint will be specified in an IVOA REC)
$$\hbox{\nolinkurl{ivo://ivoa.net/std/vibvo#biblink-harvest-0.1}}.$$
To discover the endpoints of such services, bibliography services would
execute a RegTAP query like

\begin{lstlisting}[language=SQL]
select ivoid, access_url
from rr.capability
  natural join rr.interface
where
  standard_id like 'ivo://ivoa.net/std/bibvo#biblink-harvest-0.%'
\end{lstlisting}

\section{Making VO Resources Citable}

While we believe that in the case of VO resources simply supplementing a
published article, a separate bibliographical record is generally not
very useful -- scientists can discover the \emph{data} in the VO
Registry, and for credit it is still more useful if they cite the
supplemented journal publication --, there are numerous VO resources
that do not have an (adequate) citable article; these include resources
publishing data from multiple independent journal publications,
observatory archives, the data centres themselves, and so on.

In these cases, it is desirable to have an independent bibliographic
record.  Since we do not want to establish ivoids as permanent
identifiers, VO publishers need to obtain a DOI for a resource if they
want to be eligible for inclusion into bibliographic services.

The then\dots~well, what then?  Perhaps a PleasePublishMe relationship
to a bespoke record?  A date with a bespoke role?

\appendix
\section{Changes from Previous Versions}

No previous versions yet.
% these would be subsections "Changes from v. WD-..."
% Use itemize environments.


% NOTE: IVOA recommendations must be cited from docrepo rather than ivoabib
% (REC entries there are for legacy documents only)
\bibliography{ivoatex/ivoabib,ivoatex/docrepo}


\end{document}
