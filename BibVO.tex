\documentclass[11pt,a4paper]{ivoa}
\input tthdefs

\lstloadlanguages{sh}
\lstset{flexiblecolumns=true,basicstyle=\ttfamily,showstringspaces=False}

\title{Bibliographic Interfaces in the Virtual Observatory}

% see ivoatexDoc for what group names to use here; use \ivoagroup[IG] for
% interest groups.
\ivoagroup{Data Curation and Preservation}

\author[http://www.ivoa.net/cgi-bin/twiki/bin/view/IVOA/MarkusDemleitner]{Demleitner, M.}
\author{Accomazzi, A.}

\editor{Demleitner, M.}

% \previousversion[????URL????]{????Concise Document Label????}
\previousversion{This is the first public release}


\begin{document}
\begin{abstract}
This note documents existing approaches to interfacing the Virtual
Observatory to bibliographic services, in particular the Astrophysics
data Service ADS, and suggests amendments to these existing practices
where necessary.
\end{abstract}


\section*{Conformance-related definitions}

The words ``MUST'', ``SHALL'', ``SHOULD'', ``MAY'', ``RECOMMENDED'', and
``OPTIONAL'' (in upper or lower case) used in this document are to be
interpreted as described in IETF standard RFC2119 \citep{std:RFC2119}.

The \emph{Virtual Observatory (VO)} is a
general term for a collection of federated resources that can be used
to conduct astronomical research, education, and outreach.
The \href{https://www.ivoa.net}{International
Virtual Observatory Alliance (IVOA)} is a global
collaboration of separately funded projects to develop standards and
infrastructure that enable VO applications.


\section{Introduction}

In addition to ``blind'' data discovery within the VO, users frequently
want to know about VO resources in their relationship to the wider
world, in particular that of scholarly publications.  Use cases we
envision are, specifically,

\begin{itemize}
\item A user of a bibliographic service has discovered a publication of
interest; they should also be directed to a VO resource that accompanies
that publication.

\item A user of a bibliographic service has discovered a publication
that has used VO datasets; they should be notified of the public
availability of these datasets.

\item A user of a VO resource wants to properly give credit; we
argue that resources published as a supplement to a scholarly
publication need a different treatment here from resources not directly
derived from one.
\end{itemize}

\section{Linking VO Resources to Articles}

The VOResource metadata schema \citep{2018ivoa.spec.0625P} contains the
content/source field, defined as

\begin{quotation}
\noindent A bibliographic reference from which the present resource is
derived or extracted.

\noindent This is intended to point to an article in the published
literature. An ADS Bibcode is recommended as a value when available.
\end{quotation}

\noindent in VOResource 1.2.  VOResource 1.3 will presumably add DOIs as a
recommended identifier type.

It is generally understood that, absent other rules, users of a VO
services should cite the referenced article when a resource has played a
significant role in some published research.  Facilitating this citation
is one major reason for having machine-readable identifiers in that
field.  Care must be taken to properly declare the \verb|source/@format|
to either \verb|bibcode| or \verb|doi|, or the queries below will miss
useful metadata.

Conversely, bibliographic services can use this information to enrich
bibliographic records with links to data sources (``D links'' in ADS
nomenclature).  To retrieve pairs of ivoid and bibcode, the following
RegTAP \citep{2019ivoa.spec.1011D} query can be used:

\begin{lstlisting}
select ivoid, source_value, source_format
from rr.resource
where source_format in ('doi', 'bibcode')
\end{lstlisting}

This query will work on any modern searchable registry\footnote{A list
of searchable VO registries can be obtained from the Registry of
Registries at \url{https://rofr.ivoa.net}.}.  Note, however, that
default match limits of the TAP services may already truncate this
response, so TAP clients should take care to pass a suitable large
MAXREC.

Without VO tooling, a script like the following will yield this data in
easily consumable CSV form:

\lstinputlisting[language=sh]{harvest.sh}

\appendix
\section{Changes from Previous Versions}

No previous versions yet.
% these would be subsections "Changes from v. WD-..."
% Use itemize environments.


% NOTE: IVOA recommendations must be cited from docrepo rather than ivoabib
% (REC entries there are for legacy documents only)
\bibliography{ivoatex/ivoabib,ivoatex/docrepo}


\end{document}
